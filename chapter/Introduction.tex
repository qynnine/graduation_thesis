\chapter{绪言}

\section{研究背景}

在软件演化的过程中,一份与代码实现一致的需求规约对软件的可维护性是至关重要的。事实上,与代码相比,需求规约所描述的系统高层视图更易于理解。并且,当系统实现的意图被隐藏的情况下,从系统的实现本身去理解系统的行为,这是困难且复杂的。然而,需求规约解释了系统实现的逻辑依据,能够使程序理解过程更简单、直接。此外,需求规约通常由自然语言文本描述,在与软件工程领域外的利益相关者进行讨论时(如功能更改),需求规约可以作为讨论的基础。

目前,对需求规约的更新仍然需要人工操作来完成,有高昂的人力与时间成本,且容易出错。在理想情况下,维护人员需要先更新需求,再对代码进行相应的更改。但在实际情况中,维护人员会面临以下两个难题:
\begin{enumerate}
  \item 如果需求规约的规模较大(例如,可能包含成百上千的内容页),维护人员要在其中人工地识别出受影响的需求,需要付出大量的时间和努力。
  \item 在结束对需求的更新之后,维护人员需要再次对代码进行影响性分析。代码包含许多架构与实现上的细节,这些细节并不会体现在需求规约中,因此,维护人员进一步分析代码以实施相应的更改。
\end{enumerate}
为了避免上述两次影响性分析(需求,代码)导致的高额成本,维护人员通常直接更新代码,而不更新需求。很快,需求规约将变得过时且失效,导致系统的可维护性降低。最终,系统将会进入“维修阶段”(servicing stage)[26],此时只可能对系统进行较少且次要的更改。

过时需求的定位是更新需求的关键环节。通过自动化的方式检测过时需求,能够帮助维护人员减少维护需求规约的成本,从而鼓励维护人员在软件演化的过程中,保持需求规约与代码的一致性。从技术角度来看,过时需求自动检测包含两部分内容:第一部分,通过比较代码更新前后的版本变化,识别有关的代码变更;第二部分,基于信息检索技术进行代码与需求的关联。然而,由于信息检索方法本身存在词汇失配(Vocabulary Mismtach)问题,其检索的精度有限(依赖于文本质量)。因此,如何提高过时需求自动检测方法的精度是当前研究的重点。

\section{研究现状}

在需求工程和软件演化领域,学者们认为维护一份与代码实现一致需求规约是一项挑战。在一次有关软件文档使用的调研中,Lethbridge等人\cite{lethbridge2003software}发现需求在实际的软件演化过程中很少被更新。即使需求会被更新,也通常发生在代码更新的数周以后。此外,在一些文献\cite{bennett2000software,mens2005challenges}中同样提及了需求过时并失效的问题。

在本节中,我们将会讨论与需求与代码的一致性维护有关的领域现状,包括需求可追踪性,过时需求自动检测等方面的内容。

我们所关注需求与代码的一致性维护,与代码变更后对需求的影响性分析及变更传播有关。为此,能够直接关联需求与代码的主要方式,就是需求可追踪性(Traceability)。需求可追踪性能够追踪一个需求使用期限的全过程,提供了需求到产品整个过程范围的明确的查阅能力,以辅助利益相关者完成相应的软件活动。其中,需求到代码的可追踪性是领域内关注的热点。在近十年中,需求可追踪性是一个非常活跃的研究领域。在已有的工作中,涉及包括需求可追踪性问题分析\cite{gotel1994analysis}, 需求可追踪性相关模型\cite{ramesh2001toward},需求可追踪性的自动化生成,追踪关系演化的管理\cite{cleland2003event},追踪关系的自动化维护\cite{mader2012towards}及追踪关系生成与管理工具\cite{hayes2007requirements}等。

尽管需求可追踪性对于软件系统演化的管理至关重要,但是它在实际中的使用仍然很有限。这主要是由于需求可追踪性的建立与维护需要高昂的人工与时间成本。为了解决这个问题,领域内提出了相应的方法以支持追踪关系的生成与维护。

主流的追踪关系生成方法是基于信息检索技术以计算实体间的相似性,并生成候选追踪关系\cite{antoniol2002recovering,hayes2006advancing,marcus2003recovering,cleland2005utilizing}。为了提高追踪关系生成方法的精度,信息检索技术被用于与机器学习\cite{cleland2010machine},用户反馈\cite{panichella2013and},动态分析\cite{eaddy2008cerberus}等方式相结合。此类方法的一个局限在于,当候选追踪关系的召回率水平较高时,获取的候选追踪关系准确率较低(引入了大量的误报)。因此,维护人员需要人工检查并确认生成的追踪关系是否正确,从而使基于信息检索的追踪关系生成方法退化为半自动的方法。

当可靠的追踪关系存在时,可以帮助减少两个实体间变更传播的成本。然而,对于需求和代码而言,在将其追踪关系用于变更传播时,存在如下两个方面的问题:(1)由于实现成本较高(尤其是当代码实体的数量较多时),需求与代码间的追踪关系通常无法获取(2)通常,一次变更传播只涉及部分需求与代码,需求与代码间完备的追踪关系对维护人员而言是过剩的。

为了解决上述问题,Charrada等人提出了基于代码变更的过时需求自动检测\cite{ben2012identifying}。首先,方法通过比较代码更新前后的版本变化,识别其中影响需求的代码变更。然后,方法从代码中提取关键词,以描述与需求有关的代码变更,并基于信息检索技术计算代码变更描述与需求的文本相似性,以发现潜在的过时需求。然而,需求描述的系统功能是由分布在系统间的代码协作完成的,变更部分只包含局部信息,与变更部分在结构上有依赖关系的上下文信息同样有价值,已有方法并没有充分考虑这一点。同时,在需求可追踪性,特征定位和概念分配等相关领域,代码的结构依赖已经被用于与代码的文本信息相结合\cite{mcmillan2009combining,panichella2013and,zhao2006sniafl,scanniello2015link},以提高领域内方法的效果。然而,上述工作在分析代码依赖关系时,认为所有的代码依赖关系是同样重要的,没有对各个依赖关系的重要性加以区分,从而无法充分发挥代码依赖关系的作用。此外,对于检测得到的过时需求,维护人员可能对与变更有关的知识把握不足,需要额外进行代码分析以获得更充分的变更信息,已有工作没有考虑推荐与变更有关的知识以辅助维护人员完成对过时需求的更新。

\section{本文工作}
在对需求可追踪性,需求更新等相关领域进行了深入研究后,为了解决上述问题,本文完成了以下工作:
\begin{enumerate}
  \item 基于代码依赖紧密度分析改进需求可追踪性生成方法。我们提出了一种代码依赖关系的紧密度分析方法,以区分依赖的重要性。同时,我们将此分析方法应用于需求可追踪性生成问题,提出了一种改进的需求到代码间追踪关系的生成方法TRICE(Traceability Recovery based on Information retrieval and ClosEness analysis),提高了追踪关系生成方法的精度,并在三个实验系统下验证了TRICE的有效性。
  \item 基于代码依赖关系的过时需求自动检测。我们在基于代码变更的过时需求自动检测的过程中,考虑了对代码依赖关系的分析,以引入与变更代码在结构上有依赖关系的上下文代码信息。此外,在验证了代码依赖关系的紧密度分析,能有效提高需求可追踪性生成方法效果的基础上,我们同样将代码依赖关系的紧密度分析应用于过时需求自动检测,并在三个研究案例下验证了该方法的有效性。
  \item 结合代码依赖关系与文本信息的过时需求半自动更新推荐。对于检测出的候选过时需求,我们结合代码的依赖关系与文本信息推荐与候选过时需求更新相关的变更代码元素(函数),以辅助维护人员完成对需求的更新,我们在两个研究案例下设计并完成实验,验证了该方法的有效性。
  \item 过时需求自动检测工具的实现。目前,领域内并没有相应的过时需求自动检测工具。因此,我们开发了过时需求自动检测工具INFORM(IdeNtiFy Outdated RequireMents)项目,并集成了我们基于代码依赖关系的过时需求自动检测与更新推荐方法。INFORM是一个开源项目,同时包含Console与GUI的版本,并在Console版本中集成了代码托管与版本控制服务GitHub的接口。
\end{enumerate}

\section{本文组织}
本文研究了基于代码依赖关系的需求与代码的一致性维护问题,以代码依赖关系的紧密度分析为切入点,在需求可本文研究了基于代码依赖关系的需求与代码的一致性维护问题,以代码依赖关系的紧密度分析为切入点,在需求可追踪性领域验证了代码依赖关系紧密度分析的有效性,并研究了代码依赖关系及依赖关系的紧密度分析如何提高过时需求自动检测方法的精度。论文总计六章,第一章作为绪论,提出了文本的研究背景及概要结构,其余五章的概要内容总结如下:

第二章主要介绍了与本文研究相关的概念与工作,包括需求可追踪性,过时需求自动检测。

第三章介绍了基于代码依赖紧密度分析改进需求可追踪性生成方法,提出了代码依赖关系的紧密度分析这一概念,并应用于需求可追踪性生成问题,并设计实验以验证方法的有效性。

第四章介绍基于代码依赖关系的过时需求自动检测与更新推荐方法。介绍如何通过代码依赖关系及其紧密度分析,以引入与变更代码在结构上有依赖关系的上下文代码信息,以提高过时需求自动检测的精度。同时,我们结合代码依赖关系与文本信息,向维护人员推荐与候选过时需求更新相关的变更代码元素。我们设计并完成实验,以说明上述方法的有效性。

第五章介绍了过时需求自动检测工具INFORM的使用与实现。

第六章是对全文工作的总结,并基于我们已有的研究内容展望未来工作。