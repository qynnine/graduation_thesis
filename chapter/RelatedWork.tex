\chapter{相关工作}

\section{需求可追踪性的相关概念}

\subsection{基本概念}

在软件工程领域,人们认为应该要追踪一个需求使用期限的全过程,编制每个需求同系统元素之间的联系文档,这样的关联关系提供了需求到产品整个过程范围的明确的查阅能力,需求可追踪性(Traceability)概念的提出,正是为了应对上述问题。1970年,需求可追踪性被列入美国国防部合同条款;1992年,经过软件工程领域内的广泛讨论被认为有助于软件开发实践;1994年,需求可追踪性有了如下的明确定义\cite{gotel1994analysis}:

定义:需求可追踪性是指,在软件生命周期中,对某一特定需求形成以及演变的追踪能力,既包括后向追踪,从需求文档确定后一直到软件发布过程中的各种制品之间的追踪(例如设计文档、代码和测试)又包括前向追踪,书写文档形式的需求追踪到需求的来源。

无论是正向追踪或是反响追踪,我们都可以用一个二维矩阵将对应关系固化。以需求与代码间的追踪关系为例,图N为对应的需求追踪矩阵。其中,列表示需求项,行表示代码项,矩阵间的标记表明对应列的需求与对应行的代码存在关联关系,既此代码负责实现该需求描述的行为,在实际的软件生产过程中,需求的粒度可以是用例(Use Case)或声明(Requirement Statement),对于面向对象的程序,代码的粒度通常是类或者函数。

\begin{table}[]
\centering
\caption{需求追踪矩阵(需求到代码)}
\label{my-label}
\begin{tabular}{@{}cccccc@{}}
\toprule
      & Requirement1 & Requirement2 & Requirement3 & ... & Requirement4 \\ \midrule
Code1 & X            &              &              &     &              \\
Code2 &              & X            &              &     &              \\
Code3 & X            &              &              &     & X            \\
Code4 &              &              & X            &     & X            \\
...   &              &              &              &     &              \\
Code1 &              & X            &              &     &              \\ \bottomrule
\end{tabular}
\end{table}

\subsection{研究现状}

Maeder等人\cite{mader2012assessing}分析了需求可追踪性对于软件维护任务的有益程度,他们通过研究案例iTrust与Gantt的实验,分析认为在需求可追踪性的辅助下,维护任务平均能够提高21\%的效率与60\%的准确度。然而,追踪关系的建立和维护需要也需要成本,只有当收益高于成本时,追踪关系的实施才有现实意义。文章\cite{mader2012assessing}同时分析了收益成本间的关系,分析显示随着软件演化,当维护任务达到一定数量时,在需求可追踪性的辅助下能够减少维护任务的总成本。

尽管如此,真实世界的软件系统中通常缺少固化的追踪关系\cite{ramesh1995implementing},这主要是由于(1)追踪关系在软件开发的过程中难以捕获。要求开发者在实现软件功能的同时建立和维护追踪关系代价较高,开发者实施追踪关系的意愿不高;(2)追踪关系在软件开发完成后难以重建。对于一个完成的软件系统,早期参与的开发者可能已经不在项目组中,或是遗忘系统的实现细节,导致重建追踪关系的困难。因此,领域内希望通过增强过程的自动化程度以辅助开发者完成追踪关系的生成。

目前领域内主流生成追踪关系的方法是从Antoniol等人\cite{antoniol2002recovering}的奠基性工作发展而来,该工作利用基于VSM(向量空间模型)的方法检索需求和代码间的追踪关系。方法核心在于利用信息检索技术估计文本间的相似度。继承这一想法的工作\cite{marcus2003recovering,cleland2005utilizing}分别选用LSI(潜在语义索引),JS(概率模型)检索需求和代码间的追踪关系,而比较三者模型发现,没有一种检索模型占有明显优势\cite{abadi2008traceability,eaddy2008cerberus}。

尽管基于信息检索的方法能够完全自动化的生成追踪关系,但是此类方法的精度(准确率,召回率)有限。这是由于基于文本匹配的信息检索存在词汇失配(Vocabulary Mismatch)的问题。代码的文本质量低,同义词问题等情况都会影响检索的精度。因此,出现了一系列的工作对基于信息检索的追踪关系生成方法进行改进。Dasgupta等人\cite{dasgupta2013enhancing}通过引入与代码相关的文档扩展以扩展文本的内容;Ali等人\cite{ali2012empirical}利用Eye-Tracking设备追踪维护人员在建立追踪时的眼球轨迹,以发现对维护人员重要的代码实体,并根据代码实体的重要性(函数>注释>变量>类)为来自代码实体的词项赋权;在 Ali等人\cite{ali2013trustrace}的另一份工作中,通过挖掘软件生产过程以提供额外的信息源(Commit,Bug Report,Mail List)参与决策
;Mahmoud\cite{mahmoud2013supporting}等人通过重构代码的方式改善了代码文本的质量,以减轻词汇失配的程度。Capobianco\cite{capobianco2013improving}等人则认为在检索时名词能够提供最为关键的信息,而形容词和副词等会带来干扰;Diaz等人\cite{diaz2013using}提出利用代码所有权的概念提高追踪关系的检索精度。

上述方法主要是通过引入额外文本信息或对文本赋权的方式以弥补原有需求、代码文本质量的不足。而代码与需求不同,除了文本信息之外,代码还包含结构信息,另一类工作通过深入挖掘代码的结构信息以提高追踪关系的检索精度。McMillan等人\cite{mcmillan2009combining}结合代码的文本和结构信息用于追踪关系的生成,在基于文本信息检索的候选追踪关系之上,利于结构信息进行追踪关系的重排序;Panichella等人\cite{panichella2013and}认为如果候选追踪关系本身的精度较低,则基于代码结构信息的重排序可能起到反效果,因此提出在用户反馈阶段使用代码结构信息。

\section{1.	基于信息检索的需求到代码间追踪关系生成技术}

在需求与软件过程的各种制品之间的关联关系中,需求与代码的关联关系引起了开发者的主要关注,这是由于需求描述了与软件系统行为有关的高层概念,而代码负责软件系统行为的实现,两者的关联关系能够帮助开发者理解程序,分析需求变更的影响性,完成包括维护任务、二次开发,代码复用在内的一系列软件活动。


\section{本章小结}

\bibliography{reference}
